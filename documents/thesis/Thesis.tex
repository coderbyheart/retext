\documentclass[11pt,a4paper]{article}
\usepackage[paper=a4paper,width=15cm,left=30mm,height=22cm]{geometry}

\usepackage{xltxtra}
\setmainfont[Mapping=tex-text]{Vollkorn}
\usepackage{setspace}
\linespread{1.0}
\setlength{\parskip}{0.5em}
\setlength{\parindent}{0em}

\usepackage[german]{babel}
\usepackage{graphicx}
\usepackage{acronym}

\bibliographystyle{plain} 
\bibdata{bibliothek}

\begin{document}
\author{Markus Tacker}
\title{Bachelor-Thesis zur Erlangung des akademischen Grades Bachelor of Science – B.Sc.}

\begin{center}

\begin{small}Hochschule RheinMain\\
Fachbereich Design Informatik Medien\\
Studiengang Medieninformatik

\vspace{1cm}

Bachelor-Thesis\\
zur Erlangung des akademischen Grades\\
Bachelor of Science – B.Sc.\end{small}

\vspace{2cm}

\begin{huge}Konzept und Entwurf eines workflowgesteuerten Systems zur Verwaltung von Texten in Medienprodukten\end{huge}

\end{center}

\linespread{1.25}

\vspace{10cm}

\begin{tabular}{@{}l l}
vorgelegt von & Markus Tacker\\
am & \today\\
& \\
Referent: & Prof. Dr. Jörg Berdux\\
Korreferent: & Prof. Thomas Steffen
\end{tabular}

\pagebreak

\section*{Erklärung gem. ABPO, Ziff. 6.4.3}

Ich versichere, dass ich die Bachelor-Thesis selbständig verfasst und keine anderen als
die angegebenen Hilfsmittel benutzt habe.

\vspace{2cm}

\begin{tabular*}{\textwidth}{@{\extracolsep{\fill}}l r@{}}
Offenbach am Main, \today & Markus Tacker
\end{tabular*}

\vspace{6cm}

\section*{Verbreitung}

Hiermit erkläre ich mein Einverständnis mit den im folgenden aufgeführten
Verbreitungsformen dieser Bachelor-Thesis:

\begin{tabular*}{\textwidth}{@{\extracolsep{\fill}}l r@{}}
Einstellung der Arbeit in die Hochschulbibliothek mit Datenträger: & nein \\
Einstellung der Arbeit in die Hochschulbibliothek ohne Datenträger: & nein \\
Veröffentlichung des Titels der Arbeit im Internet: & ja \\
Veröffentlichung der Arbeit im Internet: & nein
\end{tabular*}

\vspace{2cm}

\begin{tabular*}{\textwidth}{@{\extracolsep{\fill}}l r@{}}
Offenbach am Main, \today & Markus Tacker
\end{tabular*}

\pagebreak

\section*{Danksagung}

\pagebreak

\tableofcontents

\pagebreak

\section{Abstract}

\section{Problem-Analyse}

\subsection{Definition}

was sind „Texte in Medienprodukten“

\subsection{Texte in Medienprodukten}

Besonderheiten, Beteiligte, Workflow

\subsection{Microsoft Office als Standard}

Analyse der Vorteile, verwendete Funktionen

\subsection{Beispiele aus der Praxis}

Die Analyse des Problems basiert auf Interviews mit Menschen, die in ihrem Arbeitsalltag regelmäßig mit Texten zu tun haben. In diesen Interviews wurden die Personen nach ihren Erfahrungen in der Projektarbeit bezüglich Texten befragt und gebeten die aus ihrer Sicht am häufigsten auftretenden Probleme zu nennen.

\subsubsection{MAN Truck \& Bus AG: Texte für mobile Vertriebssoftware}

Markus Rüb ist als Projektleiter bei der MAN Truck \& Bus AG mit der Einführung von Tablet PCs als Vertriebshilfsmittel betraut.

\subsection{Schlussfolgerung}

\section{Konzeption eines an die spezifischen Probleme angepassten Workflows}

\subsection{Vorraussetzung / Abgrenzung}

\subsection{Workflow}

Beschreibung des optimalen Workflows und die Rolle der Beteiligten

\subsection{Beschreibung der notwendigen Funktionalität}

Unterteilung in Muss- und Kann-Kriterien

\subsection{Nachteile/Risiken des Konzepts}

\subsection{Personas}

Vorstellung (basierend auf Interviews mit realen Personen), Analyse des Konzepts in Bezug auf Personas

\subsubsection{Texter}

\section{Entwurf einer Anwendung}

\subsection{Schnittstellen}

Anforderungen, Umfang, Ausprägung für Import-, Export- und Benachrichtigungsschnittstellen

\subsection{Grundüberlegung zu einer GUI}

Anforderungen, Grundsätze, Usability, Aufbau, Wireframes

\section{Implementierung des Konzepts}

\subsection{Abgrenzung}
\subsection{Beschreibung der gewählten Umsetzung, Komponenten}

\subsection{Anwendung der Umsetzung am Beispiel des Studiengangsflyers}

\section{Fazit}

\pagebreak

\bibliography{bibliothek}

\end{document}