\documentclass[11pt,a4paper]{article}
\usepackage[paper=a4paper,width=15cm,left=30mm,height=22cm]{geometry}

\usepackage{xltxtra}
\setmainfont[Mapping=tex-text]{Bitter}
\usepackage{setspace}
\linespread{1.20}
\setlength{\parskip}{0.5em}
\setlength{\parindent}{0em}

\usepackage[german]{babel}
\usepackage{graphicx}
\usepackage{acronym}

\begin{document}
\author{Markus Tacker}
\title{Konzept Bachelorthesis}

\begin{center}

\begin{small}Fachbereich Design Informatik Medien\\Hochschule RheinMain · Bachelor-Studiengang Medieninformatik\end{small}

\bigskip

\begin{huge}Themenvorschlag Bachelor-Thesis\end{huge}

\bigskip

\begin{large}Markus Tacker\end{large}

\begin{small}\today\end{small}

\end{center}

\section*{Problemstellung}

Fast jedes Multimedia-Produkt enthält Text, denn Text ist im Gegensatz zu Grafiken, Fotos oder Animationen ein eindeutiger Informationsträger und unterliegt viel weniger stark einer Intepretation durch den Rezipienten eines Mediums als die symbolisierte oder stilisierte Darstellung von Informationen in audiovisuellen Medien. Auch aus rechtlichen Aspekten ist Text aus den genannten Gründen der einzige verbindliche Informationsträger --- bestes Beispiel hierfür ist das sogenannte „Kleingedruckte“, dass sich gerade bei inhaltlich sehr stark komprimierten Werbeformen, wie z.B. Plakat- oder Fernsehwerbung, findet.

Ist die Textmenge, die im Absatzmarketing zum Einsatz kommt, noch überschaubar, gibt es doch Medienprodukte, die ohne Text überhaupt nicht funktionieren würden. Hierunter fallen klassische Druckerzeugnisse wie Broschüren und Kataloge oder Produkte der Unternehmenskommunikation wie Jahresberichte und Pressemeldungen. Doch besonders digitale Medienprodukte werden oft mit großen Textmengen versehen --- von der einfachen Produkt-Microsite, über Werbmittel wie Newsletter bis zur Unternehmenswebseite --- die Möglichkeit Inhalte hierarchisch zu struktuieren und sogar über eine Suche zugänglich zu machen hebt eine Limitierung des Umfanges, wie bei Druckprodukten, auf. 

Alle genannten Produkte haben gemeinsam, dass ihre Erstellung in der Regel die Zusammenarbeit vieler Personen erforderlich macht. Beobachtet man den Prozess, kann man feststellen, dass es sechs verschiedene Rollen rund um die Texterstellung gibt, die ein Mitarbeiter einnehmen kann, zum Teil übernimmt eine Person dabei auch die Aufgaben mehrerer Rollen:
\begin{enumerate}
\addtolength{\itemsep}{-0.5\baselineskip}
\item Der \textbf{Informationsarchitekt} legt die Struktur eines Produktes fest und damit auch die Art und Menge des benötigten Textes,
\item der \textbf{Texter} verfasst die Texte,
\item der \textbf{Übersetzer} überträgt die Texte in weitere Sprachen,
\item der \textbf{Qualitätsmanager} überwacht die Ergebnisse der Prozesse,
\item der \textbf{Produktbesitzer} (Kunde) ist für die fachlichen und rechtliche Aspekte, sowie das Festlegen der Zeitlichen Rahmenbedingungen verantwortlich,
\item der \textbf{Produzent} ist für die Erstellung des eigentlichen Produktes verantwortlich.
\end{enumerate}
Alle Rollen haben im Verlauf eines Projekts, zu unterschiedlichen Zeiten und mit unterschiedlichem Gewicht, Einfluss auf die Gestaltung der Texte. Es existieren auch Abhängigkeiten zwischen den Rollen, so kann ein Übersetzer erst arbeiten, wenn der Text vorliegt und vom Produktbesitzer abgenommen wurde; wird aber zu einem späteren Zeitpunkt der Text geändert, muss auch wieder der Übersetzer neu beginnen.

Neben den menschlichen Einflüssen gibt es auch projektbedingte Einflüsse auf Text. Zum einen gibt es Situationen in denen in Texten bestimmte Informationen enthalten sind, die einen zeitlichen Aspekt abbilden. Ein Bespiel sind Gewinnspiele: Verschieben sich durch Probleme während dem Projekt die Zeiten, ab wann ein Produkt beim Reziepienten vorliegt, müssen auch evtl. knapp kalkulierte Gewinnspieltermine angepasst werden. Des weiteren gibt es oft erst gegen Ende eines Projekts Textänderungen aus der Rechtsabteilung des Kunden, da aus zeitlichen und finanziellen Gründen Anwälte gerne erst dann konsultiert werden, wenn Projekte kurz vor der Fertigstellung stehen. Da es Kunden von den Office-Produkten her gewöhnt sind, mit Text umzugehen, und sie aus eigener Erfahrung vermeintlich wissen „dass Texte schnell geändert sind“, erwarten sie auch, dass die Texte im Produkt bis zum Schluss geändert werden können.

So komplex auch die Abläufe bei der Erstellung von Texten für Medienprodukte sind, um so erstaunlicher ist die Tatsache, dass das Werkzeug der Wahl zur Abbildung dieser Prozesse in den allermeisten Fällen \emph{Microsoft Word} oder \emph{Excel} ist. Auf den ersten Blick bilden diese Werkzeuge viele der benötigten Funktionen rund um die Textprozesse ab, aber im alltäglichen Gebrauch treten viele Probleme gerade im Bereich des gemeinsamen Bearbeitens, paralleler oder nachträglicher Änderungen und der Übertragung der fertigen Texte in den Produktionsprozess auf. Der Grund für die Wahl der Office-Produkte liegt auf der Hand: sind sie doch in allen Unternehmen der Standard zur Textverarbeitung und sogar plattformunabhängig verfügbar --- zumindest existiert die Möglichkeit das Microsoft Office-Dateiformat auf allen Platformen zu bearbeiten. Zum Anderen existiert keine dedizierte Lösung, die explizit die genannten Abläufe in der Textverarbeitung abbildet. Es existieren vielen Produkte aus dem Bereich der Projektverwaltungswerkzeuge, Mediendatenbanken oder Content-Management-Systemen die die Prozesse rund um die Erstellung von Medienprodukten vereinfachen, aber keine kann die genannten Probleme und Abläufe zufriedenstellend abbilden.

\section*{Ziel der Arbeit}

Ziel der Bachelor-Thesis ist die Konzeptionierung einer webbasierten Anwendung, zur Unterstützung bei der Erstellung von Texten für Medienprodukte und die Implementierung einer solchen Anwendung in Form eines Prototypens. Mit der Anwendung soll es dabei mit Hilfe eines rollenbasierten Rechtemodells möglich sein, den oben skizzierten Workflow abzubilden. Jedes Medienprodukt und die dafür benötigten Texte werden dazu definiert, können mit Texten befüllt und übersetzt werden. Während dieser Vorgänge besteht jederzeit die Möglichkeit Korrekturen und Anmerkungen zu Texten zu hinterlassen. Generell werden alle Änderungen protokolliert und ein Undo bzw. Redo ist jederzeit möglich. Das System ist dabei konsequent auf den parallelen Mehrbenutzerbetrieb ausgelegt, so dass gleichzeitig, sofern die jeweiligen Vorbedingungen erfüllt sind, an allen Aspekten eines Projekts gearbeitet werden kann. Neben der \emph{Definition} und \emph{Erstellung} der Texte und ihrer \emph{Übersetzungen} ist eine weitere wichtige Funktion der Anwendung der \emph{Export} der Texte in für die Produktion nützliche Formate. Mit dem Prototypen soll es möglich sein, Texte in Form eines \emph{PDF-Textbooklets}, sowie mit Hilfe einer \emph{API} in ein systemunabhängiges XML-Format für Texte (\emph{XLIFF}\footnote{http://docs.oasis-open.org/xliff/xliff-core/xliff-core.html}) und ein sprachspezifisches Format (z.B. \emph{Java Properties}) zu exportieren. 

Um dieses Zeil zu erreichen soll eine Anwendung auf Basis eines MVC-Frameworks erstellt werden (z.B. \emph{Symfony2}\footnote{http://symfony.com/}). Zur Speicherung der Texte und der Änderungs-Historie soll eine Dokumenten-Datenbank eingesetzt werden (z.B. \emph{Apache CouchDB}\footnote{http://couchdb.apache.org/}). Die Umsetzung der Benutzeroberfläche erfolgt in \emph{HTML5} und \emph{JavaScript}, wobei der Datenaustausch vor allem mit Hilfe einer \emph{RESTful-API} erfolgt, die auch für den Export der Texte verwendet wird. Neben klassischen Unit-Tests lässt sich das gesamte System mit Hilfe der API auch leicht, unabhängig von der verwendeten Präsentationsschicht, testen.

Anhand einer einfachen GUI-Applikation (z.B. einem Musik-Player) soll die Verwendung des Systems exemplarisch  demonstriert werden.

\end{document}