\section{Fazit}\label{l:fazit}

Diese Bachelor-Thesis hat sich mit der Fragestellung beschäftigt, wie es dazu kommt, dass trotz aller technischer Fortschritte im Bereich der Informationstechnologie bei der Organisation von Texten für Medienprodukte auf Arbeitsweisen zurückgegriffen wird, die inzwischen überholt sein sollten und wie eine bessere Lösung für diese Aufgaben aussehen könnte.

Es wurde gezeigt, dass die gebräuchlichen Werkzeuge, \trademark{Microsoft Word} und \trademark{Excel}, in der alltäglichen Arbeit in Agenturen zu vielerlei Problemen führen, sie aber verwendet werden, weil die Nutzer zum einen deren Gebrauch gewöhnt sind und zum anderen die Werkzeugen scheinbar über alle notwendigen Funktionen für diese Aufgabe verfügen. In einer ausführlichen Analyse wurde diese Annahme jedoch widerlegt und im Einzelnen gezeigt, welche problematischen Auswirkungen der Einsatz monolithische Dateiformate und dezentraler Speicherung in den komplexen Abläufen in Zusammenhang mit der Erstellung von Medienprodukten haben.

Aufbauend auf dieser Erkenntnis und unter Zuhilfenahme von Personas, die auf Interviews mit Branchenexperten basieren, wurde eine Lösung konzipiert, die versucht, die gennanten Probleme zu beseitigen und den Anforderungen der Personas zu genügen. Hierzu wurde ein zentraler Anwendungsserver vorgeschlagen, mit dem die Texte für Projekte in, an die jeweiligen spezifischen Bedürfnisse angepassten, GUIs Definiert, Geschrieben, Korrigiert, Kontrolliert, Freigegeben und Veröffentlicht werden können.

Für die wichtigsten Bestandteile der Lösung, den Anwendungsserver und das browserbasierten GUI, wurde die konkrete Architektur entworfen und detaillierte Gestaltungsrichtlinen mithilfe von Wireframes festgelegt. 

Zur Validierung des Entwurfs wurde schließlich ein Prototyp umgesetzt, der die wichtigsten Funktionen anhand eines Beispiel-Projektes implementiert. Die Implementiert hat gezeigt, dass das Konzept funktioniert, der Entwurf realisierbar ist und bietet bereits in der prototypischen Fassung konkreten Mehrwert.

\secbar

Diese Bachelor-Thesis liefert eine konkrete Empfehlung für die Realisierung einer Lösung, mit der sich durch die maximale Orientierung an den Abläufen in Projekten zur Erstellung von Informations- und Kommunikations-Medien und den Bedürfnissen der beteiligten Personen in großem Maße Zeit einsparen und Fehler vermeiden lassen.

\pagebreak