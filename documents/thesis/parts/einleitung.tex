\section{Einleitung}

Die Herstellung von Informations- und Kommunikationsmedien ist heute ein zentraler Wirtschaftsbereich in Deutschland. Zahlreiche Unternehmen und Abteilungen haben es sich zur Aufgabe gemacht, Produkte zu erschaffen, die Werbebotschaften transportieren oder Information zugänglich machen. Die technischen Möglichkeiten sind vielfältig wie nie. Videos in High-Definition auf Tablet-PCs über mobile Datenverbindungen zu streamen gehört inzwischen zum Alltag. Trotz allem Fortschritt können auch die modernsten Kommunikationsformen auf einen klassischen Bestandteil nicht verzichten: \emph{Text}. Denn gerade die Medienvielfalt mit der Möglichkeit durch Multimedia-Inhalte, Interaktion und Personalisierung den Rezipienten auf emotionaler Ebene anzusprechen schafft Raum für Interpretation -- die einer Klärung bedarf, um Mißverständnisse zwischen Sender und Empfänger auszuschließen. Diese Funktion übernehmen \typoquotes{Sternchentexte} und das sprichwörtliche \typoquotes{Kleingedruckte}. Text ist also weiterhin Kernbestandteil aller Medien, von der Beschriftung eines Buttons in einer Smartphone-App bis hin zur Einladung für die nächste Aktionärsvollversammlung. Als universeller und verbindlicher Informationsträger ist es aber gerade dieser Bestandteil eines Mediums, der im Verlauf dessen Erstellung von den meisten Personen beeinflusst wird. Da die Erstellung von Medien-Produkten heute hohes technisches und fachliches Know-How erfordert, sind an diesen Projekten viele Personen beteiligt. Neben den beteiligten Mitarbeitern im Unternehmen des Auftraggebers haben auch die Mitarbeiter der beauftragen Werbeagenturen und Softwarehäusern ihre eigenen speziellen Anforderungen an die Texte des Produktes. Die stetig zunehmende Komplexität der Medien-Produkte und der Wunsch nach Kostenreduktion trägt ihren Teil dazu bei, dass der Kreis derer, die an der Erstellung eines Medien-Produktes beteiligt sind, weiter wächst: Spezialisten sind für bestimmte Aspekte des Produkts zuständig und verstärken bei Bedarf als freie Mitarbeiter das Team. Der Austausch zwischen allen Parteien über diese Anforderungen und die durch die Texte zu vermittelnden Inhalte hat sich trotz aller technischer Fortschritte in den letzten 15 Jahren nicht verändert: er findet fast ausschließlich durch das Versenden von \trademark{Word-} und \trademark{Excel-}Dateien als Anhang von E-Mails statt -- ein Prozess, der fehleranfällig, langwierig und chaotisch ist. 

\bigskip

Die Suche nach einer Möglichkeit, diesen Prozess zu verbessern ist die Motiviation für diese Bachelor-Thesis. 

\bigskip

Zu diesem Zweck wird nach der Definition wichtiger Begriffe im nächsten Kapitel zunächst in Kapitel \ref{l:problemanalyse} · S.\pageref{l:problemanalyse} analysiert, welche Probleme in branchenüblichen Projektabläufen auftreten. Aufbauend auf dieser Analyse und unter Zuhilfenahme von Personas, die in Kapitel \ref{l:personas} · S.\pageref{l:personas} vorgestellt werden, wird in Kapitel \ref{l:konzeption} · S.\pageref{l:konzeption} eine Lösung konzipiert, die versucht, die genannten Probleme zu beseitigen und den Anforderungen der Personas zu genügen. Für die wichtigsten Bestandteile der Lösung, den Anwendungsserver und das browserbasierte GUI, wird in Kapitel \ref{l:entwurf} · S.\pageref{l:entwurf} die konkrete Architektur entworfen und detaillierte Gestaltungsrichtlinen mithilfe von Wireframes festgelegt. Dieser Entwurf wird schließlich als Prototyp umgesetzt, der die wichtigsten Funktionen anhand eines Beispiel-Projektes implementiert.

Die Thesis schließt mit dem Fazit in Kapitel \ref{l:fazit} · S.\pageref{l:fazit}.

\pagebreak