\section{Einleitung}

Die Herstellung von Informations- und Kommunikationsmedien ist heute ein zentraler Wirtschaftsbereich in Deutschland. Hundertausende Unternehmen und Abteilungen haben es sich zur Aufgabe gemacht, Produkte zu erschaffen, die Werbebotschaften transportieren oder Information zugänglich zu machen -- dank des Siegeszug des Internets kann diese heute überall und zu jeder Zeit geschehen. Die technischen Möglichkeiten sind vielfältig wie nie. Videos in High-Definition auf Tablet-PCs über mobile Datenverbindungen zu streamen gehört inzwischen zum Alltag. Trotzdem allem Fortschritt können auch die modernsten Kommunikationsformen auf klassischen Bestandteil nicht verzichten: \emph{Text}. Denn gerade die Medienvielfalt mit der Möglichkeit durch Multimedia-Inhalte, Interaktion und persönlich Ansprache den Rezipienten auf emotionaler Ebene anzusprechen schafft Raum für Interpretation -- die einer Klärung bedarf, um Mißverständnisse zwischen Sender und Empfänger auszuschließen. Diese Funktion übernehmen \typoquotes{Sternchentexte} und das sprichwörtlichen \typoquotes{Kleingedruckte}. Text ist also weiterhin Kernbestandteil aller Medien, von der Beschriftung eines Buttons in einer SmartPhone-App bis hin zur Einladung für die nächste Aktionärsvollversammlung. Als universeller und verbindlicher Informationsträger ist es aber gerade dieser Bestandteil eines Mediums, der von den meisten Personen beeinflusst wird. Da die Erstellung von Medien-Produkten heute hohes technisches und fachliches Know-How erfordert, sind an diesen Projekten viele Personen beteiligt. Neben den beteiligten Mitarbeitern im Unternehmen des Auftraggebers haben auch die Mitarbeiter der beauftragen Werbeagenturen und Softwarehäusern ihre eigenen speziellen Anforderungen an die Texte des Produktes. Die stetig zunehmende Komplexität der Medien-Produkte und der Wunsch nach Kostenreduktion trägt ihren Teil dazu bei, dass der Kreis derer, die an der Erstellung eines Medien-Produktes beteiligt sind weiter wächst: Spezialisten sind für bestimmte Aspekte des Produkts zuständig und verstärken bei Bedarf als freie Mitarbeiter das Team. Der Austausch zwischen allen Parteien über diese Anforderungen und die durch die Texte zu vermittelnden Inhalte hat sich trotz aller technischer Fortschritte in den letzten 15 Jahren nicht verändert: er findet fast ausschließlich durch das Versenden von \trademark{Word-} und \trademark{Excel-}Dateien als Anhang von E-Mails statt -- ein Prozess, der fehleranfällig, langwierig und chaotisch ist. 

\bigskip

Die Suche nach einer Möglichkeit, diesen Prozess zu verbessern ist die Motiviation für diese Bachelor-Thesis. 

\pagebreak