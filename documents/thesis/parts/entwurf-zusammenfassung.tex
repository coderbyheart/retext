\subsection{Zusammenfassung}

In diesem Kapitel wurde das zentrale System der Anwendung, bestehend aus Server und browserbasierten GUI entworfen und beschrieben. Der Aufbau des Servers mithilfe einer lose gekoppelten Architektur ermöglich einfache Erweiterbarkeit und Wartung. Das browserbasierte GUI ist als JavaScript-MVC-Anwendung entworfen, die über eine REST-API mit dem Server kommuniziert. Die wichtigsten Ansichten des GUIs wurden mit Hilfe von Wireframes beschrieben. 

Der Entwurf wurde mithilfe einer prototypischen Implementierung, in der die wichtigsten Abläufe abgebildet werden, anhand eines realen Projekts überprüft. Dabei wurde gezeigt, dass sich die vorgeschlagenen Prinzipien, in der Darstellung wie in der Implementierung mit aktuell verfügbaren Technologien problemlos umsetzen lassen. Zudem wurde gezeigt, dass das Konzept der Trennung zwischen Anwendungsserver und GUI keine negativen Auswirkungen auf die Verwendbarkeit einer Anwendung hat. Für die Realisierung weiterer Funktionen im Sinne des Entwurfes sind keine Hindernisse aufgetreten.

Der in diesem Kapitel vorgestellte Entwurf liefert damit die Basis für die mögliche Entwicklung einer vollwertigen Lösung und bietet für einzelne Bestandteile bereits mögliche Technologieempfehlungen.

\secbar

Damit kann diese Bachelor-Thesis im nächsten und letzten Kapitel \ref{l:fazit} mit dem Fazit abgeschlossen werden.