Diese Bachelor-Thesis liefert eine konkrete Empfehlung für die Realisierung einer Anwendung zur Verwalten von Texten für Medienprodukte. Sie orientiert sich dabei an den tatsächlichen Abläufen in Projekten zur Erstellung von Informations- und Kommunikationsmedien und den Bedürfnissen der beteiligten Personen. Die vorgestellte Lösung bietet die Möglichkeit, im Projektverlauf in großem Maße Zeit einzusparen und Fehler zu vermeiden.

\secbar

Die Herstellung von Informations- und Kommunikationsmedien ist heute ein zentraler Wirtschaftsbereich in Deutschland. Zahlreiche Unternehmen und Abteilungen haben es sich zur Aufgabe gemacht, Produkte zu erschaffen, die Werbebotschaften transportieren oder Information zugänglich machen. Unabhängig vom Medium -- ob gedruckt oder digital -- ist \emph{Text} ein Bestandteil, der sich in fast allen Produkten wiederfindet. Als universeller und verbindlicher Informationsträger ist es aber gerade dieser Bestandteil, der von den meisten Personen beeinflusst wird. Da die Erstellung von Medien-Produkten heute hohes technisches und fachliches Know-How erfordert, sind an diesen Projekten viele Personen beteiligt. Der Austausch zwischen allen Parteien hat sich trotz aller technischen Fortschritte in den letzten 15 Jahren nicht verändert: er findet fast ausschließlich durch das Versenden der Texte als E-Mail-Anhang statt -- ein Prozess der fehleranfällig, langwierig und chaotisch ist. 

Die Suche nach einer Möglichkeit diesen Prozess zu verbessern ist die Motiviation für diese Bachelor-Thesis. 

Zu diesem Zweck wird gezeigt, dass die Werkzeuge, die am häufigsten zur Erfassung, zur Organisation und zum Austausch über diese Texte verwendet werden -- namentlich \trademark{Microsoft Word} und \trademark{Excel} -- in der alltäglichen Arbeit zu vielerlei Problemen führen. In einer ausführlichen Analyse wird die irrtümliche Annahme widerlegt, diese Werkzeuge seien für den Prozess geeignet und im Einzelnen gezeigt, welche problematischen Auswirkungen der Einsatz monolithischer Dateiformate und dezentraler Speicherung in den komplexen Abläufen in Zusammenhang mit der Erstellung von Medienprodukten haben.

Aufbauend auf dieser Erkenntnis und unter Zuhilfenahme von Personas, die auf Interviews mit zwölf Branchenexperten basieren, wird eine Lösung konzipiert, die versucht, die genannten Probleme zu beseitigen und den Anforderungen der Personas zu genügen. Hierzu wird ein zentraler Anwendungsserver vorgeschlagen, auf den mit spezialisierten, an die jeweiligen Bedürfnisse der Benutzer angepassten, GUIs zugegriffen wird. Mit deren Hilfe werden die Texte der Produkte definiert, geschreiben, korrigiert, kontrolliert, freigegeben und veröffentlicht.

Für die wichtigsten Bestandteile der Lösung, den Anwendungsserver und das browserbasierte GUI, wird die konkrete Architektur entworfen und detaillierte Gestaltungsrichtlinen mithilfe von Wireframes festgelegt.

Zur Validierung des Entwurfs wird schließlich ein Prototyp umgesetzt, der die wichtigsten Funktionen anhand eines Beispiel-Projekts implementiert. Die Implementierung zeigt, dass das Konzept funktioniert, der Entwurf realisierbar ist und bereits die prototypische Fassung konkreten Mehrwert bietet.