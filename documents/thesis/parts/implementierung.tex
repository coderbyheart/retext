\section{Implementierung eines Prototypen}\label{l:implementierung}

\subsection{Grundüberlegung}

Content-Management-Systeme bzw. Redaktionssystem können einen Teil der Aufgabe abbilden, sind aber i.d.R. ungeeignet (z.B. kein Workflow), keine Context-Informationen hinterlegbar.

\subsection{Abgrenzung}
\subsection{Beschreibung der gewählten Umsetzung, Komponenten}

\subsection{Anwendung der Umsetzung am Beispiel des Studiengangsflyers}

Die vorgeschlagene Lösung wird anhand eines realen Projektgs auf ihre praxistauglichkeit hin überprüft. Es handelt sich dabei um die einmal im Jahr erscheinende Informationsbröschüre des Studienganges Medieninformatik an der Hochschule RheinMain. Die Bröschure zu Beginn des Wintersemesters 2011/2012 einen Umfang von 28 Seiten zuzüglich Titel und Rückseite. In ihr findet sich das Grußwort des Studiengangsleiters, eine Kurzinfo über den Studiengang, das Studienprogramm mit Informationen zum Verlauf des Studiums, ein Terminkalender, Informationen zu Einrichtungen des Fachbereiches, eine Liste mit Personen im Fachbereich, sowie eine Umgebungskarte und ein Gebäudeplan. Die Broschüre wird von den Mitarbeitern des Fachbereiches selber erstellt.

