\subsection{Implementierter Workflow am Beispiel des Studiengangsflyers}

Die in den vorangegangenen Abschnitten vorgestellte Implementierung des Prototypen wird anhand eines realen Projekts auf ihre praxistauglichkeit hin überprüft. Es handelt sich dabei um die einmal im Jahr erscheinende Informationsbroschüre des Studienganges Medieninformatik an der Hochschule RheinMain. Die Broschüre zu Beginn des Wintersemesters 2011/2012 einen Umfang von 28 Seiten zuzüglich Titel und Rückseite. In ihr findet sich das Grußwort des Studiengangsleiters, eine Kurzinfo über den Studiengang, das Studienprogramm mit Informationen zum Verlauf des Studiums, ein Terminkalender, Informationen zu Einrichtungen des Fachbereiches, eine Liste mit Personen im Fachbereich, sowie eine Umgebungskarte und ein Gebäudeplan. Die Broschüre wird von den Mitarbeitern des Fachbereiches selber erstellt.

\bigskip

Mit dem Prototypen ist es möglich, den Flyer als Projekt anzulegen. Im nächsten Schritt wird die Struktur des Flyers festgelegt. Die Anwendung unterscheidet dazu zwischen Containern und Textbausteinen. Container sind rein strukturelle Elemente, die wiederum weitere Container und Textbausteine enthalten können. Für den Flyer bietet es sich an, einzelnen Kapitel (Titel, Vorwort, Kurzinfo, Termin \& Öffnungszeiten, Personen, Orientierung) als Container auf oberster Ebene anzulegen und daruntere jeweils weitere Unterteilungen nach Abschnitten vorzunehmen. Innerhalb dieser Abschnitte werden dann die einzelnen Textbausteine definiert. Zu den Textbausteinen können auch Zusatzinformationen zum Text-Typ (Schriftart, Schriftgröße, ein- oder mehrzeilig) hinterlegt werden. Die Elemente lassen sich mittels Drag\&Drop in der Reihenfolge anpassen.

Sobald die Bestandteile des Produktes angelegt wurden kann parallel bereits mit der Erstellung der Texte begonnen werden. Diese können in den jeweiligen Bausteinen je nach Typ als ein- oder mehrzeiliger Text hinterlegt werden. Änderungen an den Inhalten werden gespeichert und sind als Änderungsverlauf abrufbar. Über eine Kommentarfunktion ist der Austausch über die Texte innerhalb der Anwendung möglich. 

In der Ansicht zur Freigabe können die einzelnen Texte überprüft und abgenommen werden. Es wird zwischen den Stati für Rechtschreibung, Inhalt und der allgemeinen Freigabe unterschieden. Bei Statusänderungen wird automatisch ein Kommentar erzeugt, dass beim Ablehnen mit Zusatzinformationen durch den Benutzer versehen werden kann. Die einzelnen Stati werden aggregiert und als gesamt-Status für das gesamte Projekt angezeigt, sowie zur Übersicht und zum schnellen Zugriff in der Projektstruktur.

Als Beispiel für die Verwendung der Projektdaten in externen Systemen lässt sich das Projekt als \emph{Content-Booklet} im HTML- oder PDF-Format exportieren. Dieses Dokument enthält alle Information in strukturierter Form, so wie sie im System angelegt wurden und listet neben den Texten auch die jeweiligen Zusatzinformationen. Mit Hilfe des \emph{Content-Booklet} können Produzenten einfach mit der neuesten Version der Texte für das Produkt versorgt werden und es liefert einen Überblick über alle Bestandteile des Produktes. Ein Ausschnitt aus diesem Booklet findet sich in Anhang \ref{l:booklet} · S.\pageref{l:booklet}.

\secbar

Der Prototyp ist in der Lage, bereits mit diesem einfachen Funktionsumfang einen praktischen Mehrwehrt für ein Produkt wie den Studiengangsflyer zu bieten. Die Texte können gemeinsam und nachvollziehbar erfasst werden, der Export als \emph{Content-Booklet} kann verschiedene Mitarbeitern bei der Erstellung des fertigen Produktes konkrete Hilfestellung bieten.

