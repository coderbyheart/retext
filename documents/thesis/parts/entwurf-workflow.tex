\subsection{Implementierter Workflow}

Innerhalb der Anwendung wird das Projekt angelegt und die dafür benötigten Textbausteine definiert. Hierbei können detaillierte Angaben zu deren Eigenschaften gemacht werden, z.B. über den Verwendungszweck oder die maximal Länge. Die einzelnen Textbausteine werden bei diesem Vorgang entsprechend dem Aufbau des Endproduktes in eine Reihenfolge gebracht und hierarchisch angeordnet. So wird eine leichte Orientierung und Zuordnung der Text zum Endprodukt möglich. 

Nachdem die benötigten Textbausteine definiert wurden, werden diese durch Texter befüllt. Für Texter stellt die Anwendung Hilfsfunktionen zur Verfügung. Dazu zählen Informationen wie Zeichenlänge und Wortanzahl und Rechtschreibkorrektur mit Wörterbuch.

Sobald die Texte hinterlegt wurden durchlaufen sie die Qualitätskontrolle durch andere Mitarbeiter des Projektes und anschließend den Freigabeprozess beim Kunden. Wurden die Texte freigegeben, können die zusammengestellten Texte in das Endprodukt übernommen werden. 

Alle Vorgänge werden innerhalb der Anwendung protokolliert und sind so für jeden Beteiligten leicht nachvollziehbar. Aufgaben können automatisch aufgrund von Änderungen erzeugt werden, oder von Mitarbeiter angelegt werden. So wird sichergestellt, dass alle Projektmitarbeiter jederzeit über ihre Aufgaben bezüglich der Texte informiert sind, bei Änderungen die verantwortlichen Mitarbeiter informiert werden. Dadurch wird es möglich auch bei Korrekturen in letzter Minute diese Änderungen gezielt und transparent zu übernehmen.

