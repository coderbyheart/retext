\subsection{Zusammenfassung, Nachteile \& Risiken des Konzepts}

In diesem Abschnitt wurde eine Anwendung konzipiert und der darin abgebildetet Workflow beschrieben. Die Konzipierung der Anwendung als Web-Anwendung, bei der alle durchführbaren Operationen über Schnittstellen abgedeckt sind, ermöglicht es, für jeden Mitarbeiter die passenden Zugangswege anzubieten. Als Hauptzugang wird der Webbrowser verwendet, so ist sichergestellt, dass alle Mitarbeiter alle Funktionen des Systems ohne zusätzliche Aufwände wie die Installation neuer Software verwenden können. Für spezielle Anwendungsfällen ist es mit Hilfe der API möglich, Plug-Ins zu entwickeln, die sich in die bevorzugten Werkzeuge der Anwender integrieren.

\paragraph{Nachteile \& Risiken} Ein Nachteil dieses Konzepts liegt in der Zentralisierung der Datenspeicherung. Da alle Daten auf einem zentralen Server verwaltet werden, ist dieser auch der \emph{Single Point of Failure}, d.h. sollte der Server ausfallen, kann kein Mitarbeiter weiterarbeiten. Für einen kommerziellen Betrieb eines solchen Systems ist es also unabdingbar, dass die Server-Infrastruktur ausfallsicher konzipiert ist. 

Das Übertragen der Daten auf einen zentralen Server kann auch zu Bedenken bei den beteiligten Unternehmen führen. Es gibt gerade bei größeren Unternehmen Vorbehalte dagegen, Informationen auf Systemen von Drittanbietern zu speichern. Hier gilt es, genau wie im Hinblick auf die Verfügbarkeit des Systems, einen vertrauenswürdigen Betreiber für die Server-Infrastruktur zu finden. Alternativ ist es jedoch problemlos möglich, das System auch \emph{In-House}, also auf Servern im Unternehmen als \emph{Appliance}, zu betreiben, wobei dann aber zusätzliche Wartungsaufwände entstehen, und damit einige Vorteile des SaaS-Modells ausgehebelt werden. 

Da alle Mitarbeiter über das Internet mit der Anwendung verbunden sind, spielt auch die Bandbreite und Verfügbarkeit einer Internetverbindung eine Rolle. Im Unternehmensbereich spielt dies aber inzwischen nur nur eine untergeordnete Rolle. Trotzdem sollten geeignete Maßnahmen ergriffen werden, die die Arbeit auch mit einer langsamen oder sogar ganz ohne eine Internetverbindung ermöglicht (Offline-Access).

Das größte Risiko dieses Konzeptes ist, dass Mitarbeiter gezwungen werden, sich von ihren bekannten Werkzeugen zu lösen. Gerade bei Mitarbeitern, die vor allem mit Textverarbeitungsprogrammen arbeiten und ansonsten kaum mit anderen Werkzeugen Kontakt haben, wird der Umstieg von der unstrukturierten Arbeitsweise in \trademark{Word} auf die, bis auf den einzelnen Text heruntergebrochene Arbeitsweise in der vorgschlagenen Anwendung, schwer fallen. Man kann aber davon ausgehen, dass für alle Beteiligten die Vorteile der Lösung erkenntlich werden und sich eine Abneigung gegen eine Änderung angestammter Arbeitsabläufe leicht abbauen lässt.
