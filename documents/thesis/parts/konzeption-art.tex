\subsection{Art der Anwendung}\label{l:loesungsart}

Das System wird als \emph{browserbasierte Web-Anwendung mit vollständiger Schnittstellen-Abdeckung} konzipiert. 

\paragraph{Browserbasierte Web-Anwendung} Diese Klasse von Anwendung verwendet einen Webbrowser als Laufzeitumgebung. Dabei stellt der Browser das GUI der Anwendung mit Hilfe von HTML, CSS und JavaScript dar, die Businesslogik und die Datenhaltung wird auf einem Server ausgeführt, mit der die GUI mithilfe einer Schnittstelle kommuniziert. War es in den letzten Jahren noch üblich, dass Fragmente des GUIs mit serverseitigen Template-Sprachen erzeugt wurden (vgl.~\cite[S.48]{dunkel2008systemarchitekturen}) hat die zunehmende Verbreitung von mobilen Clients ein Umdenken zur Folge. Zum einen stellen Desktop-Clients, mobile Browser-Clients und native Apps zwar die gleichen Daten eines Systems dar, verwenden dafür aber nicht zwangsläufig die gleiche GUI-Technologie. Zum anderen werden Clients immer leistungsstärker, selbst Einsteiger-SmartPhones haben inzwischen CPUs mit mindestens dreistelligem Megahertz-Wert. Diese Entwicklung führt gerade bei Web-Anwendungen, auch Rich Internet Applications (RIAs) genannt, zu der Idee, Architekturen zu entwickeln, bei denen serverseitig keine GUI-Komponenten mehr erzeugt werden (vgl.~\cite{maccaw2011javascript}). Clients kommunizieren über Schnittstellen mit dem Server und tauschen nur noch reine Daten aus. Dies hat mehrere Vorteile. Zum einen muss serverseitig kein Modell der clientseitigen Darstellung verwaltet werden, zum anderen verkleinert sich die Menge der transferierten Daten zwischen Client und Server erheblich. Dies hat besonders bei Benutzern mit langsamen oder schlechten Datenverbindungen im Mobilfunk-Netz große Vorteile. Für Webanwendungen bedeutet dass diese das zur Darstellung benötigte HTML mit Hilfe von JavaScript selber direkt im Client erzeugen. Beim ersten Besuch einer Internetseite müssen lediglich einmal die JavaScript-Dateien und benötigte statische Ressourcen wie CSS-Dateien, Bilder und ein statischer HTML-Grundaufbau geladen werden. Anschließend werden nur noch die für die jeweilige Aktion benötigten Daten mit Hilfe von JavaScript zwischen der Anwendung und dem Server ausgetauscht. Mobile Endgeräte, die über eigene GUI-Toolkits verfügen, oder Software von Drittanbietern können dann die selben Schnittstellen verwenden, ohne dass serverseitige Anpassungen vorgenommen werden müssen. 

Web-Anwendung haben den Vorteil, dass sie ohne Installation auf dem Rechner des Benutzers lauffähig sind. Sie können als unmittelbar verwendet werden. Kompaitibilätsprobleme mit alten Browser-Versionen (z.B. dem \trademark{Internet Explorer 6}) können inzwischen mit Hilfe des \trademark{ChromeFrame}\footnote{\url{https://developers.google.com/chrome/chrome-frame/}} komfortable umgangen werden. Der Umfang an frei verfügbaren Bibliotheken zur Erstellung attraktiver und angenehm benutzbarer Anwendungen auf Basis von HTML ist riesig. Web-Anwendungen können mit wenig Aufwand auch auf mobilen Endgeräten eingesetzt werden, da Technologien zur platformabhängigen Anpassung der Darstellung (z.B. CSS-Mediaqueries) existierten. Insgesamt sind Webbrowser der aktuellen Generation mächtige Werkzeuge zur Erstellung von CRUD-Anwendungen. \cite{ms-key-software-development-trends}

\paragraph{Schnittstellen} Die Verwenden einer einheitlichen Schnittstelle durch alle Clients ermöglicht ein konsistentes Verhalten der Anwendungen über alle Zugangswege hinweg und ist besonders in Fall dieser Anwendung von Bedeutung, da die Benutzer des Systems wünschen, dass sich die Texte direkt innerhalb ihrer bevorzugten Werkzeuge abrufen und einbinden lassen. Dies ist nur mit Hilfe von Plugin-Ins möglich, die in der jeweiligen Umgebung der Software entwickelt werden müssen. Aus diesem Grund ist es ungvermeidlich, dass für alle Funktionen des Systems eine öffentliche Schnittstelle existiert.

Als Protokoll zur Kommunikation zwischen Clients und Server hat sich REST bewährt. Die Struktur des Protokolls ist direkt mit dem HTTP-Protokoll vebunden, so ist die Verarbeitung von REST-Anfragen serverseitig leicht mit Web-Frameworks zu implementieren, da diese von sich aus bereits für diese Art von Anfragen ausgelegt sind. Clientseitig wird lediglich ein HTTP-Client benötigt sowie Module zum Parsen von JSON- oder XML-Datenstrukturen -- Voraussetzungen, die von Browsern und SmartPhones erfüllt werden. JSON hat im Vergleich zu SOAP den Vorteil, dass es nicht versucht die Architektur der zugrundeliegenden Software nach außen abzubilden, so muss sich der Client nicht an bestimmte Reihenfolgen im Aufruf von Methoden halten. In der REST-Welt sind alle Operationen atomar und können ohne Vorbedingung gestellt werden. In der Praxis ist dies nicht immer umsetzbar, REST fordert serverseitig Zustandslosigkeit, die aber bei Systemen in denen Daten gespeichert und verändert werden nicht realisierbar ist. Aufgrund seines flexibleren Aufbaus, der Möglichkeit ausgewählte Anfragen leicht mit HTTP-Caches zu beschleunigen und der freien Wahl der Nachrichtenformats ist REST aus sicht des Autors die besser Wahl zur Implementierung der Schnittstellenkommunikation.