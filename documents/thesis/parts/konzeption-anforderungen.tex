\subsection{Anforderungen}\label{l:anforderungen}

Wie in der Schlussfolgerung in Abschnitt~\ref{l:schlussfolgerung} bereits erwähnt ergeben sich aus den genannten Problemen im vorangegangenen Kapitel die folgenden Anforderungen an eine Lösung.

\paragraph{Gleichzeitiges Bearbeiten von Texten} Es soll möglich sein, dass alle  Mitarbeiter gleichzeitig an den Texten eines Produktes arbeiten.



\subsection{Eigenschaften von Texten}
\label{l:textattribute}

\paragraph{Typ} Überschrift, Untertitel, Bild-Beschreibung, Fließtext.

\subsection{Beschreibung der notwendigen Funktionalität}

Unterteilung in Muss- und Kann-Kriterien

\subsection{Nachteile/Risiken des Konzepts}

\subsection{Funktionale Anforderungen}


\TODO

Aufteilen der Texte in einzelne Bausteine um diese eindeutig identifizieren zu können. Dies verhindert Copy\&Paste-Fehler (vgl. S.~\pageref{p:serielles-konzept}).

\label{l:hierarchien} Hierarchien sind aber in allen Produkten vorhanden und ein natürlicher Weg, Informationen zu gliedern. 

Fallback-Texte
