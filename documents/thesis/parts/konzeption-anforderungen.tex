\subsection{Anforderungen an die Anwendung}\label{l:anforderungen}

Neben den im vorangegangenen Abschnitt beschriebenen Anforderungen an den Workflow, muss die Anwendung weitere Eigenschaften erfüllen. Die wichtigsten werden nachfolgend kurz beschrieben.

\paragraph{Gleichzeitiges Bearbeiten}

Es ist essentiell, die Beschränkungen eines seriellen Bearbeitungskonzeptes, wie es durch \trademark{Word} und \trademark{Excel} vorgegeben wird, aufzuheben und das gleichzeitige Bearbeiten der Inhalte des Projektes zu ermöglichen. 

\paragraph{Benutzerverwaltung und Nachvollziehbarkeit}

Die Abbildung eines Workflows setzt voraus, dass die Beteiligten am Projekt eindeutig identifiziert werden können und dass es möglich ist, die Aufgaben explizit zu verteilen. Das ist auch Voraussetzung für eine weitere Anforderungen nach der Nachvollziehbarkeit aller Änderungen.

\paragraph{Zugang von überall}

Da sich die Projektbeteiligten nicht alle an einem Ort befinden, muss der Zugang zur Anwendung über das Internet möglich sein.

\paragraph{Zugriff von verschiedenen Plattformen}

Es gibt unterschiedliche Anforderungen an den Zugang zur Plattform, dieser muss möglichst von vielen Endgeräten und dabei sowohl von stationären Systemen als auch von unterwegs aus möglich sein.

\paragraph{Integration in die Werkzeuge}

Für viele Arbeiten in Zusammenhang mit der Erstellung von Medien werden spezialisiert Werkzeuge verwendet, die durch die hier vorgestellte Lösung niemals ersetzt werden können. Es muss also eine Integration in diese Werkzeuge mit Hilfe von Plug-Ins oder ähnlichem existieren, mit denen es möglich wird, die Texte aus der Anwendung in die Produkte zu übernehmen. Zur Anbindung dieser Plug-Ins wird eine Schnittstelle (API) benötigt.

\paragraph{Umfangreiche Export-Funktionen}

Um die Texte in das Produkt zu integrieren, bedarf es umfangreicher Exportfunktionen in strukturierete Formate wie z.B. XML. Zur Kontrolle oder um eine Übersicht über das Produkt zu bekommen ist es erforderlich den Export in Dokumentenformate wie \trademark{PDF} und \trademark{Word} zu ermöglichen.

