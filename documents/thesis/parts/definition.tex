\section{Definition}
\label{l:def}

In dieser Bachelor-Thesis werden bestimmte allgemeine Begriffe und deren Synonyme verwendet, deren konkrete Bedeutung im Kontext dieser Arbeit wie folgt definiert ist:

\paragraph{Workflow (Ablauf)} Die Automatisierung eines Business-Prozesses, als Ganzes oder in Teilen, in welchem Dokumente, Informationen oder Aufgaben entsprechend einer Menge von prozeduralen Regeln von einem zum anderen Teilnehmer zur Bearbeitung weitergegeben werden \cite[S.8]{wmc}. Allgemein lässt sich sagen, dass ein Workflow aus den zum Erreichen eines Zieles nötigen Arbeitsschritten besteht.

\paragraph{Text (Textbaustein)} Damit sind die kleinsten sinnvoll identifizierbaren Bestandteile gemeint, aus denen sich der Text eines Produktes zusammensetzt. Dies sind in der Regel einzelne Sätze bei Druckmedien, können aber auch einzelne Worte sein, wie z.B. die Beschriftung einer Schaltfläche in einer Anwendung.

\paragraph{Medium (Medien-Produkt, Produkt)} Medien sind physische oder elektronische Informationsträger. Diese Bachelor-Thesis beschäftigt sich vor allem mit Informations- und Kommunikationsmedien und hierbei vor allem mit Massenkommunikationsmitteln, sowohl physischer als auch technischer Natur (vgl. \cite[S.199--201]{schanze2002metzler}). Dies können z.B. Marketingmedien wie Broschüren oder Fernsehwerbespots sein aber auch softwarebasierte Produkte wie eine Smartphone-Anwendungen oder eine Internetseiten. 

\paragraph{Agentur} Ein Unternehmen das Medien erstellt. In der Regel sind dies Werbeagenturen, Medien-Produktionsfirmen oder Software-Systemhäuser. 

\paragraph{Projekt} Die Erstellung von Medien erfolgt in Agenturen in Projektarbeit. Projekte sind zeitlich begrenzt und vereinen zielgerichtet die zur Erstellung des Produktes beteiligten Mitarbeiter und Ressourcen. 

\paragraph{Kunde} Ein Unternehmen das Agenturen mit der Erstellung von Medien beauftragt.

\paragraph{Nutzer} Eine Person, die ein Medium konsumiert oder ein Produkt verwendet.

\paragraph{Werkzeug (Anwendung)} Eine Software, die eine spezielle Funktion erfüllt. Adobe Photoshop ist ein Werkzeug zur Bearbeitung von Bildern.

\paragraph{Wireframes} Wireframes dienen zur schnellen Skizzierung eines Produktes mit Hilfe von einfachen geometrischen Formen und Texten. Sie werden angefertigt um einen Überblick über den gesamten Umfang eines Produktes zu bekommen die wichtigsten Aufgaben zu definieren, die von dessen Nutzern durchgeführt werden.