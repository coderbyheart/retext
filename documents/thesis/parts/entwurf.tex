\section{Entwurf}\label{l:entwurf}

Diese vier Leitlinien repräsentieren die Grundgedanken bei der Entwicklung von der Anwendung:

\begin{itemize}
\item{Das wichtigste zuerst: Die aktuelle Aufgabe soll immer im Fokus der Darstellung liegen.}
\item{Schnell zum Ziel: Alle Aufgaben müssen leicht und umkompliziert durchführbar sein.}
\item{Nicht nerven: Ständige Benachrichtigungen lenken ab und müssen deswegen so gestaltet sein, dass diese sich nach den Präferenzen des Nutzers richten.}
\item{Hilfe nur einen Klick entfernt: Das Hilfesystem muss kontextsensitiv verfügbar sein und ist eine Kernfunktion der Anwendung}
\end{itemize}

\subsection{Komponenten}

\subsubsection{Core}

\subsubsection{Persistenz}

\subsubsection{Schnittstellen}

\subsubsection{Benachrichtigung}

\subsubsection{Pull-Export}

\subsubsection{Push-Export}

\subsection{Browserbasierte GUI}

Anforderungen, Grundsätze, Usability, Aufbau, Wireframes

Bei Kontroll-Aufgaben (Lektorat, QS) unterbrechungsfreies Arbeiten ermöglichen (Infinite-Scroll).

Die GUI muss deutlich einfacher zu bedienen sein, als z.B. Word oder Publishing-Systeme, sonst wird sie nicht von Kunden eingesetzt.

\subsubsection{Wireframes}