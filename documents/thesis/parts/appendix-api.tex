\section{API-Endpunkte}\label{l:api-endpoints}

\subsection{Benutzer}

\begin{tabular}{@{}l l l}
\textbf{Methode} & \textbf{Pfad} & \textbf{Beschreibung}\\
\hline\\[-1.5ex]
\texttt{POST} & \texttt{/user} & Benutzer registrieren\\
\texttt{POST} & \texttt{/login} & Login\\
\texttt{POST} & \texttt{/logout} & Logout\\
\texttt{GET} & \texttt{/auth} & Sitzungsstatus abfragen\\
\texttt{GET} & \texttt{/user/<id>} & Benutzerprofil laden\\
\end{tabular}

\subsection{Projekt}

\begin{tabular}{@{}l l l}
\textbf{Methode} & \textbf{Pfad} & \textbf{Beschreibung}\\
\hline\\[-1.5ex]
\texttt{POST} & \texttt{/project} & Neues Projekt anlegen\\
\texttt{GET} & \texttt{/project} & Liste mit Projekten des Nutzers laden\\
\texttt{GET} & \texttt{/project/<id>} & Projekt laden\\
\texttt{GET} & \texttt{/project/<id>/progress} & Fortschritt des Projektes laden\\
\end{tabular}

\subsection{Elemente}

\begin{tabular}{@{}l l l}
\textbf{Methode} & \textbf{Pfad} & \textbf{Beschreibung}\\
\hline\\[-1.5ex]
\texttt{GET} & \texttt{/element} & Liste mit Elementen (Container, Text) unterhalb eines Elementes laden\\
\end{tabular}

\subsection{Container}

\begin{tabular}{@{}l l l}
\textbf{Methode} & \textbf{Pfad} & \textbf{Beschreibung}\\
\hline\\[-1.5ex]
\texttt{POST} & \texttt{/container}& Container anlegen\\
\texttt{GET} & \texttt{/container} & Liste mit Containernunterhalb eines Elementes laden\\
\texttt{GET} & \texttt{/container/<id>} & Container laden\\
\texttt{PUT} & \texttt{/container/<id>} & Container aktualisieren\\
\texttt{DELETE} & \texttt{/container/<id>} & Container löschen\\
\texttt{GET} & \texttt{/container/<id>/breadcrumb} & Navigationspfad ab einem Container bis zum\\
&&obersten Container laden\\
\texttt{GET} & \texttt{/container/<id>/tree} & Baumstruktur ab einem Container laden\\
\end{tabular}

\subsection{Texte}

\begin{tabular}{@{}l l l}
\textbf{Methode} & \textbf{Pfad} & \textbf{Beschreibung}\\
\hline\\[-1.5ex]
\texttt{POST} & \texttt{/text} & Text anlegen\\
\texttt{GET} & \texttt{/text/<id>} & Text laden\\
\texttt{PUT} & \texttt{/text/<id>} & Text aktualisieren\\
\texttt{DELETE} & \texttt{/text/<id>}\\
\texttt{POST} & \texttt{/text/<id>/comments} & Kommentar zu einem Text anlegen\\
\texttt{GET} & \texttt{/text/<id>/comments} & Kommentare zu einem Text laden\\
\texttt{GET} & \texttt{/text/<id>/history} & Änderungshistorie zu einem Text laden\\
\end{tabular}

\subsection{Text-Typen}

\begin{tabular}{@{}l l l}
\textbf{Methode} & \textbf{Pfad} & \textbf{Beschreibung}\\
\hline\\[-1.5ex]
\texttt{GET} & \texttt{/texttype} & Liste mit Text-Typen eines Projektes laden\\
\texttt{GET} & \texttt{/texttype/<id>} & Text-Typ laden\\
\texttt{PUT} & \texttt{/texttype/<id>} & Text-Typ aktualisieren\\
\end{tabular}

\subsection{Export}

\begin{tabular}{@{}l l l}
\textbf{Methode} & \textbf{Pfad} & \textbf{Beschreibung}\\
\hline\\[-1.5ex]
\texttt{GET} & \texttt{/export/contentbooklet.pdf} & Content-Booklet eines Projektes\\
&&als PDF exportieren\\
\texttt{GET} & \texttt{/export/contentbooklet.html} & Content-Booklet eines Projektes\\
&&als HTML exportieren\\
\end{tabular}

\pagebreak