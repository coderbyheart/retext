\subsection{Anbindung der GUI an den Server}\label{l:anbindung-gui}

Die in Abbildung~\ref{chart:komponenten} · S.\pageref{chart:komponenten} \emph{gelb} eingezeichneten Komponenten zeigen den Aufbau eines browserbasierten GUIs. Mittlerweile ist es üblich, klassische Paradigmen aus der Softwarentwicklung auch in browserbasierten Anwendung zu verwenden. Dementsprechend wird die GUI in Form einer MVC-Anwendung implementiert. Die Kommunikation mit dem Server wird in einer eigenen Komponente, dem \emph{API-Adapter}, gekapselt, die über die REST-Schnitt"-stelle JSON-Daten mit dem Server austauscht. Die Repräsentation der Domänendaten erfolgt dabei mithilfe von entsprechenden \emph{Models}, die durch die \emph{Controller} mit den \emph{Views} verbunden werden. Die Models sind nicht zwangsläufig mit den Models auf der Serverseite identisch sondern können Aggregationen sein, oder nur Teile der Daten abbilden, sie orientieren sich an der öffentlichen API des Servers, die nicht zwingenderweise die interenen Models 1:1 nach außen weiterreicht. Views sind einzelne Bestandteile einer Ansicht. So lassen sich individuelle Bereiche der Darstellung leicht aktualisieren, ohne die ganze Seite neu laden zu müssen -- in \cite[S.1--5 und S.65--72]{maccaw2011javascript} ist dieses Prinzip ausführlich erläutert.

Ein Problem beim Datenaustausch mit JSON ist, dass diese üblicherweise nur reine Daten enthält, da schlichterweise Objekte serialisiert werden. Beim Serialisieren werden dann nur die Eigenschaften einer Objekt-Instanz beibehalten, Informationen wie die Klasse oder Relationen zu anderen Objekten werden verworfen. Werden APIs nach diesem Schema verwendet, müssen die Entwickler wissen, hinter welchem Endpunkt sie welche Arten von Daten erwarten. Diese Information wird als Quellcode hart kodiert. 

Um zukünftigen Änderungen an der Schnittstelle ohne Änderungen auf Clientseite begegnen zu können sollten die Antworten der Schnittstelle mit Meta-Informationen versehen sein, die es dem Client ermöglichen, ohne hart kodierte Zuordnung zwischen Endpunkt und Domänenobjekt auszukommen. \trademark{JSON-LD}\footnote{\url{http://json-ld.org/}} hat sich dabei als Lösung bewährt. Hierbei werden JSON-Objekt mit Meta-Informationen versehen, die beschreiben, um \emph{was} für ein Objekt es sich bei der Antwort handelt (\texttt{@context}) und um \emph{welches} (\texttt{@id}):

\begin{verbatim}
{
  "@context": "http://json-ld.org/contexts/person.jsonld",
  "@id": "http://dbpedia.org/resource/John_Lennon",
  "name": "John Lennon",
  "born": "1940-10-09",
  "spouse": "http://dbpedia.org/resource/Cynthia_Lennon"
}
\end{verbatim}

In dieser Art kann ein JSON-Objekt auch mit Informationen über zugehörige \emph{andere} Objekte versehen werden. Im Beispiel verweist \texttt{spouse} auf die URL unter der das JSON-Ibjekt der Ehefrau abgerufen werden kann. Auf diese Weise lassen sich die vollständige Datenstruktur einer Anwendung als Graph darstellen. Hierzu werden alle Objekte mit Relationen annotiert und der Client kann entscheiden, welchen Relationen er folgt und welchen nicht. Die die Relations-Informationen auch die URLs zu den Endpunkten enthalten, müssen in der Implementierung keine Endpunkte mehr hart codiert werden, es ist lediglich nötig, einen Einstiegs-Punkt zu kennen von dem aus man mit sich durch den Graphen \typoquotes{hangeln} kann, indem man Objekt-Relationen folgt, deren Kontexte die Anwendung erkennt.\footnote{\url{http://coderbyheart.de/blog/relationen-in-linked-data}}