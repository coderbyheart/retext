\section*{Zusammenfassung}

\TODO

Nahezu alle Informations- und Kommunikationsmedien haben eines gemeinsam: sie beinhalten Text. Obwohl viele Personen bei der Erstellung dieser Texte beteiligt sind, werden sie in der Regel in Office-Dokumenten verwaltet, meistens mit Word, bei großen Projekten kommt Excel zu Einsatz. Der Workflow von einem Bearbeiter zum nächsten erfolgt über den Austausch des Office-Dokumentes via E-Mail, Ticketsysteme oder sonstige System zum asynchronen Dateiaustausch wie z.B. Dropbox. Dieser Prozess ist aufwendig und fehleranfällig. Sobald mehrere Personen gleichzeitig an den Texten arbeiten, wird manuelles Eingreifen notwendig um die gemachten Änderungen zusammenzuführen. Aufgrund der Vielzahl der am Text beteiligten Personen sind Office-Dateien ein denkbar schlecht geeignetes Mittel um Texte und ihre Änderungen sauber und nachvollziehbar zu verwalten. Auch das Übertragen von Texten aus Office-Dokumenten ist eine Fehlerquelle – es ist stupides Copy\&\-Paste. In den meisten Fällen müssen dabei im Dokument vorgenommene Formatierungen wie Umbrüche und Absätze entfernt werden um eine saubere Darstellung im Endprodukt zu gewährleisten. Gerade der Text ist der Bestandteil eines Informations- und Kommunikationsmediums, der oft bis zur letzten Minute geändert wird – egal wie viel Aufwand vorher in die Planung geflossen sind. Dies liegt unter anderem daran, dass Text im Gegensatz zu Grafiken, Fotos und anderen Multimedia-Elementen als einziger Informationsträger eindeutig ist und üblicherweise keinen Interpretationsspielraum offen lassen soll. So bietet er auch aus rechtlicher Sicht den problematischsten Bestandteil des Produkts.

Diese Bachelor-Thesis analysiert das beschriebene Problem und konzipiert einen passgenauen Worflow, der alle Beteiligten entsprechend ihrer Aufgabe und Anforderungen integriert. Als Proof-of-Concept wird eine webbasiertes Anwendung entworfen, die den konzipierten Workflow soweit abbildet, dass das Konzept am Beispiel einer Informationsbroschüre überprüft werden kann. 

Im Abschnitt \ref{l:entwurf} wird eine web-basierte Anwendung entworfen, die diesen Anforderungen entspricht. Im Abschnitt \ref{l:personas} werden schließlich Personas vorgestellt, die typische Benutzer der Lösung repräsentieren und zur Überprüfung des Konzeptes verwendet werden.

\pagebreak